% vim: ts=2: sw=2: expandtab: autoindent: spell:
%%%%%%%%%%%%%%%%%%%%%%% file typeinst.tex %%%%%%%%%%%%%%%%%%%%%%%%%
%
% This is the LaTeX source for the instructions to authors using
% the LaTeX document class 'llncs.cls' for contributions to
% the Lecture Notes in Computer Sciences series.
% http://www.springer.com/lncs       Springer Heidelberg 2006/05/04
%
% It may be used as a template for your own input - copy it
% to a new file with a new name and use it as the basis
% for your article.
%
% NB: the document class 'llncs' has its own and detailed documentation, see
% ftp://ftp.springer.de/data/pubftp/pub/tex/latex/llncs/latex2e/llncsdoc.pdf
%
%%%%%%%%%%%%%%%%%%%%%%%%%%%%%%%%%%%%%%%%%%%%%%%%%%%%%%%%%%%%%%%%%%%


\documentclass[runningheads,a4paper]{llncs}
\usepackage[T1]{fontenc}
\usepackage{amssymb}
\usepackage{amsmath}
\setcounter{tocdepth}{3}
\usepackage{graphicx}
\usepackage{hyperref}
\usepackage[hang,labelfont=bf]{caption}
\usepackage[normalem]{ulem}
\usepackage{xcolor}
\makeatletter
\def\squiggly{\bgroup \markoverwith{\textcolor{red}{\lower3.5\p@\hbox{\sixly \char58}}}\ULon}
\makeatother
\usepackage[square,numbers]{natbib}


\newcommand{\keywords}[1]{\par\addvspace\baselineskip
\noindent\keywordname\enspace\ignorespaces#1}

\newcommand{\Fw}{\ensuremath{\mathrm{F}_\omega}}
\newcommand{\Fi}{\ensuremath{\mathrm{F}_i}}
\newcommand{\HMX}{\ensuremath{\mathrm{HM}(\mathcal{X})}}

\begin{document}

\mainmatter  % start of an individual contribution

% first the title is needed
\title{A Case Study on Polymorphic Type Inference using Prolog}

% a short form should be given in case it is too long for the running head
\titlerunning{A Case Study on Polymorphic Type Inference using Prolog}

% the name(s) of the author(s) follow(s) next
%
% NB: Chinese authors should write their first names(s) in front of
% their surnames. This ensures that the names appear correctly in
% the running heads and the author index.
%
\author{Ki Yung Ahn\and Andrea Vezzosi
%% \thanks{Please note that the LNCS Editorial assumes that all authors have used
%% the western naming convention, with given names preceding surnames. This determines
%% the structure of the names in the running heads and the author index.}%
%% \and Ursula Barth\and Ingrid Haas\and Frank Holzwarth\and\\
%% Anna Kramer\and Leonie Kunz\and Christine Rei\ss\and\\
%% Nicole Sator\and Erika Siebert-Cole\and Peter Stra\ss er
}
%
\authorrunning{A Case Study on Polymorphic Type Inference using Prolog}
% (feature abused for this document to repeat the title also on left hand pages)

% the affiliations are given next; don't give your e-mail address
% unless you accept that it will be published
\institute{Portland State University,\\ Portland, OR, USA\\
     \path|kya@pdx.edu| \\~\\
     Chalmers University of Technology,\\ Gothenburg, Sweden\\
     \path|vezzosi@chalmers.se|
   }

%
% NB: a more complex sample for affiliations and the mapping to the
% corresponding authors can be found in the file "llncs.dem"
% (search for the string "\mainmatter" where a contribution starts).
% "llncs.dem" accompanies the document class "llncs.cls".
%

\toctitle{A Case Study on Polymorphic Type Inference using Prolog}
\tocauthor{A Case Study on Polymorphic Type Inference using Prolog}
\maketitle


\begin{abstract}
A concise, declarative, and machine executable specification of
the Hindley--Milner type system (HM) can be formulated using
logic programming languages such as Prolog. Modern functional
language implementations such as the Glasgow Haskell Compiler
supports more extensive flavors of polymorphism beyond Milner's
theory of type polymorphism in late '70s. In this case study,
we progressively extend the Prolog specification to include
more advanced type system features. An interesting development is
that extending dimensions of polymorphism beyond HM resulted in
a multi-staged solution: resolve the typing relations first,
while delaying to resolve kinding relations, and then resolve
the delayed kinding relations. Our case study shows that Prolog is
an effective tool for prototyping type inference with rich features
of polymorphism, and that logic programming could have been even more effective
for specifying type inference if it were equipped with better theories and
tools for staged resolution of different relations at different levels.
\keywords{Hindley--Milner, functional language, type system,
  type inference, unification, parametric polymorphism,
  higher-kinded polymorphism, type constructor polymorphism,
  kind polymorphism, algebraic datatype, nested datatype,
  logic programming, Prolog, delayed goals
%%         GADT,
%%         generalized algebraic datatype
        }
\end{abstract}


%%%%%% \section{Introduction}\label{sec:intro} %%%%%%%%%%%%%%%
\section{Introduction}\label{sec:intro}
When implementing the type system of a programming language,
we often face a gap between the design described on paper and
the actual implementation. Sometimes there is no formal description
but only an ambiguous description in English (or in other natural languages).
Even when there is a mathematical description of the type system on paper,
there can be a gap between the type system design and implementation.
Language designers and implementers can suffer from this gap because
it is difficult to determine whether a problem originated from a flaw
in the design or a bug in the implementation. Having a declarative
(i.e., structurally similar to the design)
and flexible (i.e., easily extensible) machine executable specification
is extremely helpful, especially in early prototyping stage or
when experimenting with possible extensions to the language type system.  
Jones' attempt of \emph{Typing Haskell in Haskell} \cite{JonesTHiH99} is
an exemplary work that demonstrates the value of such a concise, declarative,
and machine executable specification:
90+ pages of Hugs type checker implementation in C code specified in
only 400+ lines readable Haskell.

In our work, we use Prolog to specify advanced polymorphic features of
modern functional languages. Logic programming languages like Prolog
are natural candidates for the purpose of specifying type systems
that support type inference: the syntax and semantics are designed to represent
logical inference rules (which is how type systems are typically formalized) and
they offer native support for unification (which is the basic building block of
type inference algorithms). As a result, Prolog specifications are usually more
succinct than a specification using a functional language.

Our contributions are:
\begin{itemize}\vspace*{-1ex}
\item A succint and declarative specification of
	a type system with several dimensions of polymorphism
	in less than 30 lines of Prolog (\S\ref{sec:poly}).
\item An easily extensible specifiation for pragmatic use
	(as demonstrated in \S\ref{sec:other}):
We believe our specification is usable without expert knowledge
in logic programming because we only used built-ins and fairly basic
library predicates, not relying on any specialized frameworks.
\item Two-staged inference for types and kinds using delayed goals
	(\S\ref{ssec:HMtck}):
We discovered that kind inference can be delayed after type inference
(it is in some sense quite natural) and exploited the fact to get the
most out of Prolog's native unification in both type and kind inference
(which helped our specification to be more succinct).
\item A Motivating example for calling support for a dual-view
	on variables in logic programming:
Type variables are viewed as unification variables
in type inference and as concrete atoimc variables
in kind inference in our specification.
Better way of organizing this idea is desirable
for even better specification.
\end{itemize}

We give a step-by-step tutorial style explanation of our specification in
\S\ref{sec:poly}, gradually extending from the Prolog specification of
the simply-typed lambda calculus. In \S\ref{sec:other}, we demonstrate
that our method of specification is flexible for extensions with
other language features. All our Prolog specification in \S\ref{sec:poly}
and \S\ref{sec:other} are tested on SWI Prolog 7.2 and its source code is
available online.\footnote{
	\url{https://github.com/kyagrd/HMtyInferUsingProlog} }
We contemplate on more challenging language features such as GADTs and
term indicies in \S\ref{sec:futwork}, discuss related work in
\S\ref{sec:relwork}, and summarize our discussion in \S\ref{sec:concl}.

%% oustidein \cite{OutsideInICFP09}


%%% TODO TODO cite these
%% delayed goals are not uncommon in logic programming
%% (AILog
%% 
%% http://artint.info/html/ArtInt_329.html
%% 
%% https://books.google.com/books?id=Q4i5edW7IyQC&pg=PA321&lpg=PA321
%%
%% https://books.google.com/books?id=d2_i0qeP6yUC&pg=PA645&lpg=PA645
%%

 %%%%%%%%%%%%%%%%%%%%%%%%%%%%%%%%%%%%%%%%%%%%%%%%
%%%%%%%%%%%%%%%%%%%%%%%%%%%%%%%%%%%%%%%%%%%%%%%%%%%%%%%%%%%%%%

\section{Polymorphic type inference specifications in Prolog}
\label{sec:poly}
We start from a Prolog specification for the type system of
simply-typed lambda calculus (STLC) (\S\ref{ssec:STLC})
and gradually add more features of polymorphism. We discuss
the specification for HM (\S\ref{ssec:HM}), which extends STLC
with type polymorphism, and then extend the specification to support
type constructor polymorphism and kind polymorphism (\S\ref{ssec:HMtck}).

%% All Prolog specifications (Figs.~\ref{fig:STLC}, \ref{fig:HM},
%% and \ref{fig:HMtck}) in this section are tested in SWI-Prolog 7.2.
The following two lines must be loaded into the Prolog system
before running the Prolog specifications.
\begin{verbatim}
  :- set_prolog_flag(occurs_check,true).
  :- op(500,yfx,$).
\end{verbatim} \noindent
In the first line, \verb|set_prolog_flag(occurs_check,true)|
sets Prolog's unification operator \verb|=| to perform occurs check.
For example, \verb|X -> Y = Y| will fail because of occurs check.
In the second line, \verb|op(500,yfx,$)| declares \verb|$|
as a left associative infix operator, which is used to represent
application operator in the object language syntax. For instance,
\verb|E1 $ E2| is an application of \verb|E1| to
\verb|E2|.

\subsection{STLC}\label{ssec:STLC}
The type system of STLC can be specified in 4 lines
(excluding empty lines) of pure\footnote{
    Prolog implementations usually support features beyond
    first-order predicate logic with unification, which are
    considered impure parts of Prolog. For example,
    control flow operations such as cut (\texttt{!/0}) or
    meta-logical operations such as variable test (\texttt{var/1}) are
    ISO standard Prolog built-ins.} Prolog (Fig.~\ref{fig:STLC}).
The \verb|type| predicate is self explanatory; its three rule
definition almost literally transcribes the three typing rules of
the STLC. For instance, the third rule
\texttt{\small type(C,X\$Y,B) :- type(C,X,A->B), type(C,Y,A)}
is a transcription of the typing rule for applications
$~~\frac{C\vdash X:A\to B\qquad~~\quad C\vdash Y:A}{C\vdash X\,Y:B}$ in STLC.
The \verb|first| predicate implements looking up a binding
(\verb|X:T|) in the typing context~(\verb|C|)
by first occurrence of the variable name (\verb|X|).
In case of repeated variable names, choosing the first occurrence in the context reflects shadowing of bound variables,
e.g. the term $(\lambda x. \lambda x. x)$ will have only one type schema, the one of $(\lambda x. \lambda y. y)$.
We assume variable names are given as atoms.\footnote{
    In Prolog, identifiers that start with an uppercase letter or
    an underscore (e.g., \texttt{X}, \texttt{Xs}, or \texttt{\_G1}) are
    variables that could be matched with other terms. Identifiers that
    startwith a lowercase letter (e.g., \texttt{x}, \texttt{xs}, or
    \texttt{i\_am\_atom}) and appear alone (i.e., not as a function symbol
    such as \texttt{f} in \texttt{f(X,Y)}), are atoms, which are
    distinct form atoms with other names or complex terms
    (e.g., \texttt{f(X,Y)}).} For example, \verb|var(x)| is
a variable expression whose name is represented by the atom \verb|x|.

\begin{figure}
\vspace*{-1ex}
\begin{verbatim}
  type(C,var(X),       T) :- first(X:T,C).
  type(C,lam(X,E),A -> B) :- type([X:A|C], E,  B).
  type(C,X $ Y,        B) :- type(C,X,A -> B), type(C,Y,A).

  first(K:V,[K1:V1|Xs]) :- K=K1 -> V=V1 ; first(K:V, Xs).
\end{verbatim}
\vspace*{-1ex}
\caption{STLC in Prolog}
\label{fig:STLC}
\end{figure}

One great merit of this specification is that it also serves as
a machine executable reference implementation. We can use Prolog
to run the specification for type checking, as shown in the following query:
\begin{verbatim}
  ?- type([], lam(x,var(x)), A -> A).         
  true .
\end{verbatim}\noindent
and also for type inference, as shown the following query:
\begin{verbatim}
  ?- type([], lam(f,lam(x,var(f)$var(x))), T).
  T = ((_G1861->_G1862)->_G1861->_G1862) .
\end{verbatim}
In the following sections, we discuss how to add polymorphic features
to the specification. The specifications with the extended features
also serves as machine executable reference implementations, which
are able to perform both type checking and type inference.


\subsection{HM}\label{ssec:HM}
HM supports rank-1 polymorphic types (a.k.a. type schemes)
in contrast to STLC, which only support monomorphic types.
Polymorphic variable bindings are introduced by let expressions in HM.
For example, the following let expression introduces a polymorphic binding
for the identity function ($\textit{id}:\forall X.X\to X$):
\[\textsf{let}~\textit{id} = (\lambda x. x)
  ~\textsf{in}~\textit{id}\;\textit{id} \; .\]
This let expression is well-typed in HM because the polymorphic type
can be instantiated differently in each occurrence of \textit{id}
in the application expression. We distinguish different occurences by superscripts:
\[\frac{\textit{id}:\forall X.X\!\to\!X\vdash
          \textit{id}^{\,\textcolor{red}{1}}\! : (\squiggly{A \to A}) \to (A \to A)
        \qquad
        \textit{id}:\forall X.X\!\to\!X\vdash
          \textit{id}^{\,\textcolor{red}{2}}\! : \squiggly{A \to A}
       }
  {\textit{id}:\forall X.X\!\to\!X\vdash
      \textit{id}^{\,\textcolor{red}{1}}\,\textit{id}^{\,\textcolor{red}{2}}
      \! : A \to A
  } \; .
\]
If \textit{id} had been bound monomorphically, the application would be ill-typed
because the type of \textit{id} would have to unify with its domain: 
\[\frac{\textit{id}:A\!\to\!A\vdash
          \textit{id}^{\,\textcolor{red}{1}}\! : \!\squiggly{\,A\,}\! \to A
        \qquad
        \textit{id}:A\!\to\!A\vdash
          \textit{id}^{\,\textcolor{red}{2}}\! : \squiggly{A \to A}
       }
  {\textit{id}:A\!\to\!A\vdash
      \textit{id}^{\,\textcolor{red}{1}}\,\textit{id}^{\,\textcolor{red}{2}}
      \! : \text{\textcolor{red}{\texttt{ERROR:} cannot unify
               $A \,\pmb{\doteq}\, A\!\to\!A$}}
  } \; .
\]

\begin{figure}
\begin{verbatim}
  type(C,var(X),      T1) :- first(X:T,C), instantiate(T,T1).
  type(C,lam(X,E),A -> B) :- type([X:mono(A)  |C], E,  B).
  type(C,X $ Y,        B) :- type(C,X,A -> B), type(C,Y,A).
  type(C,let(X=E0,E1), T) :- type(C,               E0, A),
                             type([X:poly(C,A)|C], E1, T).
  
  instantiate(poly(CXT,T),T1) :- copy_term(t(CXT,T),t(CXT,T1)).
  instantiate(mono(T),    T).
  
  first(K:V,[K1:V1|Xs]) :- K=K1 -> V=V1 ; first(K:V, Xs).
\end{verbatim}
\caption{HM in Prolog}
\label{fig:HM}
\end{figure}

In the specification of HM (Fig.~\ref{fig:HM}), a monomorphic type
is represented as \verb|mono(A)|, which corresponds to simply \verb|A|
in the STLC specification (Fig.~\ref{fig:HM}), whereas a polymorphic type
has the form \verb|poly(C,A)|. A typing context in HM may contain
two kinds of bindings: monomorphic bindings (\verb|X:mono(A)|) and
polymorphic bindings (\verb|X:poly(C,A)|).

The four rules defining the \verb|type| predicate are almost literal
transcriptions of the typing rules of HM. The first rule for finding
a type (\verb|T1|) of a variable expression (\verb|var(X)|) amounts
to instantiating (\verb|instantiate(T,T1)|) the type (\verb|T|)
in the binding (\verb|X:T|) that first matches the variable
bound in the typing context (\verb|C|).
The two rules for lambda and application expressions are essentially
the same as the corresponding rules in STLC.
The last rule for let expressions introduce polymorphic bindings.
Note that a polymorphic biding \verb|poly(C,A)| refers to
the typing context \verb|C| of the let expression.

The \verb|instantiate| predicate
cleverly implements the idea of polymorphic instantiation in HM.
The built-in predicate \verb|copy_term| makes a copy of the first argument
and unifies it with the second argument. The copied version is identical to
the original term except all the Prolog variables have been substituted with
freshly generated variables. The instantiation a polymorphic type
\verb|poly(CTX,T)| is implemented as \verb|copy_term(t(CTX,T),t(CTX,T1))|.
Firstly, a copied version of \verb|t(CTX,T)| is made. Say \verb|t(CTX2,T2)|
is the copied version with all variables in both \verb|CTX| and \verb|T|
are freshly renamed in \verb|CTX2| and \verb|T2|. Secondly, \verb|t(CTX2,T2)|
is unified with \verb|t(CTX,T1)|, which amounts to \verb|CTX2=CTX| and
\verb|T2=T1|. Because \verb|CTX2| is being unified with the original
context \verb|CTX|, all freshly generated variables in \verb|CTX2| are
unified with the original variables in \verb|CTX|. Therefore, only
the variables in \verb|T| that do not occur in its binding context
\verb|CTX| shall effectively be freshly instantiated in \verb|T1|.
This exactly captures generalization and instantiation of
polymorphic types in HM.

The Prolog specification of HM is only 8 lines (excluding empty lines).

\subsection{HM + Type Constructor Polymorphism + Kind Polymorphism}
\label{ssec:HMtck}
Modern functional languages such as Haskell support rich flavours of
polymorphism beyond type polymorphism. For example, consider
a generic tree datatype
\[ \textbf{data}~\textit{Tree}~c~a
  ~=~ \textit{Leaf}\,~a ~\mid~ \textit{Node}~(c~(Tree~c~a)) \]
where $c$ determines the branching structure at the node and $a$
determines the type of the value hanging on the leaf. For instance,
the tree datatype above instantiates to a binary tree
when $c$ instantiates to a pair constructor and
a rose tree when $c$ instantiates to a list constructor.
The type system of Haskell infers that $c$ has kind $*\to*$ and
$a$ has kind $*$. That is, $c$ is a unary type constructor
that takes an argument of kind $*$ and $a$ is type. The type system of
Haskell is also able to infer types for polymorphic functions defined
over \textit{Tree}s, which may involve polymorphism over type constructors,
such as $c$, as well as over types, such as $a$.

The Prolog specification in Fig.~\ref{fig:HMtck} implements
type constructor polymorphism and kind polymorphism, in only 26 lines
(excluding empty lines).
We get kind polymorphism for free because we can reuse
the same \verb|instantiate| predicate for kinds as well as types;
this is a delightful side-effect of Prolog not being statically typed.
Here, we focus our discussion on the modifications to support
type constructor polymorphism.

Supporting type constructor variables of
arbitrary kinds introduces the possibility of ill-kinded type (constructor)
formation (e.g., $F\,G$ when $F:*\!\to\!*$ but $G:*\!\to\!*$ \;or\; 
$A\!\to\!B$ when $A:*\!\to\!*$). In our Prolog specification, we use
the atomic symbol \verb|o| to represent the kind usually notated with $*$
(e.g., in Haskell) because \verb|*| is predefined as a built-in infix
operator in Prolog. The \text{kind} predicate transcribes the kinding rules
for well-formed kinds: a type constructor variable must be bounded in
the kinding context (\verb|KC|), a function type \verb|A->B| is
a well-formed type (of kind \verb|o|) when both \verb|A| and \verb|B|
are well-formed types (of kind \verb|o|), and a type constructor application
\verb|F$G| is well-formed when \verb|F| is an arrow kind (\verb|K1->K2|)
whose codomain matches with the kind (\verb|K1|) of its argument \verb|G|.
A type that satisfy these three kinding rules are well-formed kinds, 
or well-kinded.

The typing rules (\verb|type|) need some modification from the rules of HM,
in order to invoke checks for well-kindedness using the kinding rules
(\verb|kind|). We discuss the modification in three steps.

First step is to have the typing rules take an additional argument for
kinding context (\verb|KC|) along with the typing context (\verb|C|).
Because the kinding rules require a kinding context. The typing rules
should keep track of the kinding context in order to invoke \verb|kind|
from \verb|type|. Thus, we add another argument to \verb|type| for
the kinding context (i.e., from \verb|type(C,...)| to \verb|type(KC,C,...)|).

\begin{figure}
\begin{verbatim}
  kind(KC,var(Z),K1) :- first(Z:K,KC), instantiate(K,K1).
  kind(KC,F $ G, K2) :- kind(KC,F,K1 -> K2), kind(KC,G,K1).
  kind(KC,A -> B,o)  :- kind(KC,A,o), kind(KC,B,o).
  
  type(KC,C,var(X),     T1,G,G ) :- first(X:T,C), instantiate(T,T1).
  type(KC,C,lam(X,E), A->B,G,G1) :- type(KC,[X:mono(A)|C],E,B,G0,G1),
                                    G0 = [kind(KC,A->B,o)|G]. % delay goal
  type(KC,C,X $ Y,       B,G,G1) :- type(KC,C,X,A->B,G, G0),
                                    type(KC,C,Y,A,   G0,G1).
  type(KC,C,let(X=E0,E1),T,G,G1) :- type(KC,C,              E0,A,G, G0),
                                    type(KC,[X:poly(C,A)|C],E1,T,G0,G1).
  
  instantiate(poly(C,T),T1) :- copy_term(t(C,T),t(C,T1)).
  instantiate(mono(T),  T).
  
  first(K:V,[K1:V1|Xs]) :- K=K1 -> V=V1 ; first(K:V,Xs).
  
  infer_type(KC,C,E,T) :-
    type(KC,C,E,T,[],Gs0), % 1st stage of typing in this line
    copy_term(Gs0,Gs),     % 2nd stage of kinding from here
    findall(Ty, member(kind(_,Ty,_),Gs), Tys),
    free_variables(Tys, Xs), % collect all free tyvars in Xs
    maplist(variablize,Xs),  % concretize tyvar with var(t) where t fresh
    findall(A:K, member(var(A),Xs), KC1), % kind binidngs for var(t)
    appendKC(Gs,KC1,Gs1),  % extend kinding context with new kind bindings
    maplist(call,Gs1).     % run all goals in Gs1
  
  variablize(var(X)) :- gensym(t,X).

  appendKC([],_,[]).
  appendKC([kind(KC,X,K)|Gs],KC1,[kind(KC2,X,K)|Gs1]) :-
    append(KC1,KC,KC2), appendKC(Gs,KC1,Gs1).
\end{verbatim}
\caption{HM + type constructor polymorphism + kind polymorphism in Prolog
        $\qquad$
        (without pattern matching).}
\label{fig:HMtck}
\end{figure}

The second step is to invoke well-kindedness checks from the necessary
places among the typing rules. We follow the formulation of Pure Type
Systems \cite{Barendregt91}, a generic theory of typed lambda calculi,
which indicates that well-kindednss checks are required at the
formation of function types, that is, in the typing rule for lambda
expressions (the second rule of \verb|type|).
A straightforward modification for this typing rule would be
\begin{verbatim}
  type(KC,C,lam(X,E),A->B) :- type(KC,C,[X:mono(A)|C],E,B),
                              kind(KC,A->B,o).
\end{verbatim}
This second step modification is intuitive as a specification, but
rather fragile as a reference implementation. For instance,
a simple type inference query for the identity function fails
\begin{verbatim}
  ?- type([],[],lam(x,var(x)),T).
  ERROR: Out of local stack
\end{verbatim}
In STLC and HM specifications, such query would have successfully
inferred its type as \verb|T = A -> A|.

There are mainly two reasons for the erratic behavior.
Firstly, there is not enough information at the moment of
well-kindedness checking. At the invocation of \verb|kind|,
the only available information is that it is a function type \verb|A -> B|.
Whether \verb|A| or \verb|B| is a variable, a type constructor application,
or a function type may be determined later on, when there are other parts
of expression to be type checked or type inferred. Secondly, we have
a conflicting view on type variables at the typing level and
at the kinding level. At the typing level, we think of type variables as
unification variables, implemented by Prolog variables in order to exploit
the unification natively supported in Prolog. At the kinding level,
on the contrary, we think of type variables as concrete names that
can be looked up in the kinding context (just like term variables
in the typing context).

The last step of the modification addresses the erratic behavior of
the second step. A solution for these two problems mentioned above is
to stage the control flow: first, get as much information as possible
at the typing level, and then, concretize Prolog variables with atomic names
for the rest of the work at the kinding level. Instead of directly invoking
\verb|kind| in the second rule of \verb|type|, we collect the list of
all the necessary well-kindedness checks in a list to be handled later.
This programming technique is known as \emph{delayed goals}
in logic programming.
You can also think of it as building up a to-do list or a continuation.
We add two additional arguments to \verb|type| to manage delayed goals.
That is, \verb|type| has six arguments, \verb|type(KC,C,E,A,T,Gs,Gs1)|.
The fifth (input) argument \verb|Gs| is the current list of delayed goals
before invoking \verb|type|. The last (output) argument \verb|Gs1| obtains
the list of delayed goals after invoking \verb|type|. These two arguments
are threaded through the recursive definition of
the \verb|type| predicate.\footnote{
  Experienced Prolog programmer would use
  \emph{DCG Grammer rules} \cite{SWIPrologManual} to abstract away
  such extra arguments for accumulated differences.
  Here, we choose to stick to basic constructs to be more accessible,
  relying less on Prolog idioms and libraries.}

The \verb|infer_type| predicate implements the two-staged solution,
as follows:
\begin{itemize}
\item[1.]
The first line of the definition is the first stage at the typing level.
For example, when inferring a type for the identity function,
\begin{verbatim}
  ?- type([],[],lam(x,var(x)),T,[],Gs0).
  T = (_G1643->_G1643),
  Gs0 = [kind([], (_G1643->_G1643), o)] .
\end{verbatim}
It infers the most generic type (\verb|_G1643->_G1643|)
of the identity function and generates one delayed goal
\verb|kind([], (_G1643->_G1643), o)|.
\item[2.] From the second line,
we invoke this delayed goal after preprocessing of concertizing
the type variables. In the second line, we make a copied version of
the delayed goals using \verb|copy_term| in order to decouple
the variables of the first stage from the variables of the second stage.
After the second line, \verb|Gs| contains a copied version of \verb|Gs0|
with freshly renamed variables, say
\verb|Gs = [kind([], (_G2211->_G2211), o)]|.
\item[3,4.]
The third and fourth line of the definition collects all the type variables
in \verb|Gs| into \verb|Xs|, that is, \verb|Xs = [_G2211]|, continuing with
our example for the identity function.
\item[5.]
The fifth line \verb|maplist(variablize,Xs)| instantiates
the Prolog variables collected in \verb|Xs| into
concrete type variables with fresh names. \verb|variablize|
implements fresh concrete variable generation using
a symbol generation library predicate \verb|gensym|.
Invoking \verb|gensym(t,X)| generates atoms whose fresh names that
start with \verb|t|. For instance, \verb|X=t1|, \verb|X=t2|, and so on.
After fifth line, \verb|Xs = [var(t1)]| and
\verb|Gs = [kind([], (var(t1)->var(t1)), o)]| because 
the \verb|variablize| predicate has instantiated \verb|_G2211|
as \verb|_G2211=var(t1)|.
\item[6,7.] Freshly generated concrete type variables need to be
registered to the kinding context before invoking \verb|kind|
because the first kinding rule requires that type variables must be
bound in the typing context. The sixth line creates monomorphic
type bindings for all the variable names in \verb|Xs|, and collects
them into \verb|KC1|. Continuing with our identity example,
\verb|KC1 = [t1:mono(K1)]| after the sixth line. The seventh line
extends each kinding context in the dealyed goal \verb|Gs| with
the bindings (\verb|KC1|) for the freshly generated variables, 
where the goals with extended contexts are collected in \verb|Gs1|.
Ater seventh line, we have 
\verb|Gs1 = [kind([t1:mono(K1)], (var(t1)->var(t1)), o)]|.
\item[8.] Finally, we perform the delayed well-kindedness checks by calling
all the goals in \verb|Gs1|. We can execute the well-kindedness check for
our identity example by querying
\begin{verbatim}
  ?- kind([t1:mono(K1)], (var(t1)->var(t1)), o).
  K1 = o
\end{verbatim}
The well-kindedness check succeeds by inferring that the freshly generated
type variable \verb|var(t1)| must have kind \verb|o|.
\end{itemize}

\section{Supporting Other Language Features}\label{sec:other}
The purpose of this section is to demonstrate that our Prolog specification
for polymorphic features is extensible for supporting other orthogonal
features in functional languages including general recursion
(\S\ref{ssec:letrec}), pattern matching over algebraic datatypes
(\S\ref{ssec:patlam}), and recursion schemes over non-regular
algebraic datatypes with user provided annotations (\S\ref{ssec:mit}).
The specification for the pattern-matching and the recursion schemes
in this section are extensions that build upon the specification
in \S\ref{ssec:HMtck}.

Discussions on the details of the Prolog code is kept relatively brief,
compared to the previous section, because our main purpose here is to
demonstrate that supporting these features does not significantly increase
the size and the complexity of our specification. Readers with further
interest are encouraged to experiment with our specifications available
online.

\subsection{Recursive Let-bindings}\label{ssec:letrec}
Adding recursive let-bindings is obvious. Simply add
a monomorphic binding for the let-bound variable (\verb|X|)
when inferring the type of the expression (\verb|E0|) defining
the let-bound value, as follows:
{\small
\begin{verbatim}
type(KC,C,letrec(X=E0,E1),T,G,G1) :- type(KC,[X:mono(A)  |C],E0,A,G, G0),
                                     type(KC,[X:poly(C,A)|C],E1,T,G0,G1).
\end{verbatim} }

We could also allow polymorphic recursion by type annotations
on the let-bound variable, like we will do for Mendler-style iteration
over non-regular datatypes \S\ref{ssec:mit}.


\subsection{Pattern Matching for Algebraic Datatypes}\label{ssec:patlam}
In Fig.\ref{fig:patlam} (on p\pageref{fig:patlam}), we specify
pattern matching expressions without the scrutinee, which is
also known as pattern-matching lambdas. A pattern lambda is
a function that awaits an expression to be passed in
as an argument to pattern-match with its value. For example,
let $\{p_1\to e_1;\cdots; p_n\to e_n\}$ be a pattern-matching lambda.
Then, the application $\,\{p_1\to e_1;\cdots; p_n\to e_n\}\,e\,$
corresponds to a pattern matching expression in Haskell
${\bf\;case}\;e\;{\bf of}\;\{p_1\to e_1;\cdots; p_n\to e_n\}$.

We represent pattern-matching lambdas in Prolog as a list of clauses
that match each pattern to a body, for instance,
\verb|['Nil'-->E1, 'Cons'(x,xs)-->E2]|
where \verb|E1| and \verb|E2| are expressions of the bodies. For simplicity,
we implement the most simple design of non-nested patterns. That is,
a pattern is either an atom that represents a nullary data constructor,
such as \verb|'Nil'|, or a complex term with $n$-ary function symbol
that represents $n$-ary data constructor and $n$ variables as arguments,
such as \verb|'Cons'(x,xs)|.
Atoms and function symbols normally start with lowercase letters in Prolog.
However, Prolog allows other names to become atoms and function symbols
when those names are single-quoted. Here, we are using the convention
such that names of type constructors and data constructors start with 
uppercase letters while names of term variables (including pattern variables)
start with lowercase letters. We also add a delayed well-kindedness goal
because patten lambdas introduce function types (\verb|A-> T|), just like
ordinary lambda expressions.

The specification above for non-nested pattern-matching lambdas
is only 16 lines. To sum up, the specification for HM extended
with type constructor polymorphism, kind polymorphism, and
non-nested pattern matching for algebraic datatypes is in $26+16=42$
lines of Prolog, which is also an executable reference implementation
of the type system.

\begin{figure}
\begin{verbatim}
type(KC,C,Alts,A->T,G,G1) :- type_alts(KC,C,Alts,A->T,G,G1),
                             G0 = [kind(KC,A->T,o) | G]. % delayed goal

type_alts(KC,C,[Alt],          A->T,G,G1) :- % assume at least one clause
  type_alt(KC,C,Alt,A->T,G,G1).              % empty list not allowed
type_alts(KC,C,[Alt,Alt2|Alts],A->T,G,G1) :-
  type_alt(KC,C,Alt,A->T,G,G0),
  type_alts(KC,C,[Alt2|Alts],A->T,G0,G1).

type_alt(KC,C,P->E,A->T,G,G1) :- % assume non-nested single depth pattern
  P =.. [Ctor|Xs], upper_atom(Ctor), % when P='Cons'(x,xs) then Xs=[x,xs]
  findall(var(X),member(X,Xs),Vs),   %    Vs = [var(x),var(xs)]  
  foldl_ap(var(Ctor),Vs,PE),         %    PE = var('Cons')$var(x)$var(xs)
  findall(X:mono(Tx),member(X,Xs),C1,C), % extend C with bindings for Xs
  type(KC,C1,PE,A,G,G0), % type of the pattern as an expression is A
  type(KC,C1,E,T,G0,G1). % type of the body is T

upper_atom(A) :- atom(A), atom_chars(A,[C|_]), char_type(C,upper).
lower_atom(A) :- atom(A), atom_chars(A,[C|_]), char_type(C,lower).
\end{verbatim}
\vspace*{-1ex}
\caption{A Prolog specification of non-nested pattern-matching lambdas
$\qquad$ (coverage checking not included).}
\label{fig:patlam}
\end{figure}
\begin{figure}
\begin{verbatim}

kind(KC,mu(F), K)  :- kind(KC,F,K->K).

type(KC,C,in(N,E),T,G,G1) :- type(KC,C,E,T0,G,G1),
                             unfold_N_ap(1+N,T0,F,[mu(F)|Is]),
                             foldl_ap(mu(F),Is,T).
type(KC,C,mit(X,Alts),mu(F)->T,G,G1) :-
  is_list(Alts), gensym(r,R),
  KC1 = [R:mono(o)|KC], C1 = [X:poly(C,var(R)->T)|C],
  type_alts(KC1,C1,Alts,F$var(R)->T,G,G1).
type(KC,C,mit(X,Is-->T0,Alts),A->T,G,G1) :-
  length(Is,N), length(Es,N), foldl_ap(mu(F),Es,A),
  is_list(Alts), gensym(r,R), foldl_ap(var(R),Is,RIs),
  KC1 = [R:mono(K)|KC], C1 = [X:poly(C,RIs->T0)|C],
  G0 = [kind(KC,F,K->K), kind(KC,A->T,o) | G], % delayed goal
  foldl_ap(F,[var(R)|Is],FRIs),
  type_alts(KC1,C1,Alts,FRIs->T,G0,G1).

unfold_N_ap(0,E,    E,[]).
unfold_N_ap(N,E0$E1,E,Es) :-
  N>0, M is N-1, unfold_N_ap(M,E0,E,Es0), append(Es0,[E1],Es).

foldl_ap(E, []     , E).
foldl_ap(E0,[E1|Es], E) :- foldl_ap(E0$E1, Es, E).
\end{verbatim}
\vspace*{-1ex}
\caption{A Prolog specification for Mendler-style iteration.}
\label{fig:mit}
\end{figure}

%% TODO Show typing rules, especially for mendler things

\subsection{Recursion Schemes for Non-Regular Algebraic Datatypes}
\label{ssec:mit}
Consider the following two recursive datatype declarations in Haskell:
\begin{align*}
& \textbf{data}~\textit{List}~\,a~
      \,=\, N_{\!L}
       ~|~  C_{\!L}~a~(\textit{List}~a)\\
& \textbf{data}~\textit{Bush}~a
      \,=\, N_{\!B}
       ~|~  C_{\!B}~a~(\textit{Bush}\;(\textit{Bush}~a))
\end{align*}
\textit{List} is a homogeneous list, which is either empty
or en element tailed by a \textit{List} that contains (zero or more)
elements of \emph{the same type as the prior element}.
\textit{Bush} is a list-like structure that is either empty
or has en element tailed by a \textit{Bush} that contains (zero or more) 
elements \emph{but their type $(\textit{Bush}~a)$ is different from 
the type of the prior element $(a)$}.
The recursive component (\textit{List a} in $C_{\!L}~a~(\textit{List}~a)$),
which is the tail of a list, has exactly the same type argument ($a$) as
the datatype being defined
(\textit{List a} in $\textbf{data}~\textit{List}~a = \ldots$).

Every recursive component of \textit{List}, which is the tail of a list,
has exactly the same type argument ($a$) as the \textit{List} containing
the tail. In other words, the types of recursive occurrences in \textit{List}
is always the same, or \emph{regular}. When all recursive occurrences in
a (recursive) datatype are regular, as in \emph{List}, it is called
a \emph{regular datatype}. The recursive component of \textit{Bush},
on the contrary, has different type argument (\textit{Bush a}) from
the type argument ($a$) of its containing \textit{Bush}. That is,
a bush headed by an element of type $a$ is tailed by a bush whose
head element is of type \emph{Bush a}. A recursive datatype that
has a non-regular recursive occurrence, as in \emph{Bush}, is called
a \emph{non-regular datatype}, also known as
a \emph{nested datatype}\;\cite{BirMee98} because the types of
recursive components typically become nested as the recursion goes deeper
(e.g., $\textit{Bush}(\textit{Bush}(\cdots(Bush(\textit{Bush}~a))\cdots))$).

In order to define interesting and useful recursive functions over
non-regular datatypes, one needs polymorphic recursion, whose 
type inference is known to be undecidable without the aid of
user supplied type annotations. In Fig.~\ref{fig:mit} (on p\ref{fig:mit}),
we specify a subset of a functional language that supports a recursion scheme,
which naturally generalize from regular datatypes to non-regular datatypes.
In particular, we specify
the Mendler-style iteration \cite{matthes98phd,AbeMatUus03}
supported in the Nax language \cite{Ahn14thesis}. In Nax,
all recursive constructs, both at the type level and at the term level,
are defined using the primitives provided by the language, avoiding
uncontrolled general recursion.

%% \begin{figure}
%% \[ (\mu)\frac{\Delta \vdash F : \kappa \to \kappa}{
%%               \Delta : \mu_\kappa F : \kappa} \]
%% 
%% \[ (\textbf{in})
%%     \frac{ \Delta;\Gamma \vdash e :  F(\mu_\kappa F)\,\overline{A}
%%            \qquad \Delta \vdash F : \kappa \to \kappa
%%            \qquad \Delta \vdash \mu_\kappa F\, \overline{A} : *}{
%%            \Delta;\Gamma \vdash
%%            \textbf{in}_\kappa\,e  : \mu_\kappa F \, \overline{A} }
%% \]
%% \[ (\textbf{mit})
%%    \frac{\Delta;\Gamma \vdash }{
%%      \Delta;\Gamma \vdash \textbf{mit}_\kappa : \}
%% \]
%% \caption{A kinding rule ($\mu$) and typing rules of Mendler-style iteration}
%% \label{fig:mitrule}
%% \end{figure}

The \verb|mu(F)| appearing in the Prolog specification
corresponds to the recursive types $\mu F$
constructed by the fixpoint type operator $\mu$ applied to
a base structure $F$, which is not recursive by itself.
Here, we require that $F$ is either a type constructor introduced by
a (non-recursive) datatype declaration or partial application of
such a type constructor. We add a kinding rule for the fixpoint type
operator by adding another rule of the \verb|kind| predicate for \verb|mu(F)|.
The Mendler-style iteration over regular datatypes (\verb|mit(X,Alts)|)
does not need any type annotation. The Mendler-style iteration over
non-regular datatypes (\verb|mit(X,Is-->T0,Alts)|) needs an annotation
(\verb|Is-->T0|) to guide the type inference because it is likely to rely on
polymorphic recursion. The specification for Mendler-style iteration relies on
pattern-matching lambdas discussed in the previous subsection. Once we have
properly set up the kinding context and typing context for the name of
recursive call (\verb|X|), the rest amounts to inferring types for
pattern-matching lambdas.
Pointers to further details on Mendler-style recursion
\cite{vene00phd,AbeMatUus03,AhnShe11} and Nax \cite{Ahn14thesis}
are available in the references section at the end of this paper.

The specification for Mendler-style iteration in Fig.~\ref{fig:mit}
is only 20 lines, including pattern-matching lambdas $16+20=36$ lines,
including HM with type constructor polymorphism and kind polymorphism
the total is $26+16+20=62$ lines of Prolog (excluding empty lines).
The complete specification, and also a reference implementation, of
all the features discussed so far fits in only 62 lines of Prolog.

A missing part from a typical functional language type system, which
we have not discussed in this paper, is the initial phase of populating
the kinding context and typing context from the list of algebraic datatype
declarations prior to type checking the expressions using them.
With fully functioning basic building blocks for kind inference
(\verb|kind|) and type inference (\verb|type|), inferring kinds of
type constructor names and inferring types for their associated
data constructors should be straightforward.


\section{Future Work}\label{sec:futwork}
We plan to continue our work on two additional features.
One feature is generalized algebraic datatypes (GADTs)
and the other is enriching the kind syntax, as in the
Nax language, to support term-indicies in datatypes,
especially in GADTs.

GADTs add complexity of introducing local constraints
within a pattern-matching clause, which should not escape
the scope of the clause, unlike global unification constraints in HM.
It would be interesting to see whether Prolog's built-in support for
handling unification variables and symbols could help us express
the concept of local constraints as elegantly as we expressed
polymorphic instantiation in \S\ref{sec:poly}. In addition,
more user supplied annotations will be required to guide the type inference,
even when recursion is not involved, because GADT type inference is
undecidable \cite{DegtyarevV95} regardless of the use of recursion.
In Nax, pattern-matching lambdas can have annotations, which are
just like the annotations on the Mender-style iteration in \S\ref{ssec:mit},
to aid type inference.

The kind structure needed for type constructor (i.e. higher-kinded) polymorphism is exactly
the kinds supported in the higher-order polymorphic lambda calculus,
known as System \Fw\ \cite{girard72thesis}. Type constructors in \Fw\ can have
types as arguments. For example, the type constructor \textit{List}
for lists has kind $* \to *$, which means that it needs one type argument
to be fully applied as a type (e.g. $\textit{List}~\texttt{Nat} : *$).
\begin{align*}
  &\text{kind in System \Fw}  &&\kappa ::= * \mid \kappa \to \kappa \\
  &\text{kind in System \Fi}  &&\kappa ::= * \mid \kappa \to \kappa
                                             \mid \{A\,\}\to \kappa
\end{align*}
To support terms, as well as types, to be applied to type constructors
as arguments, the kind structure needs to be extended.
System~\Fi\ \cite{AhnSheFioPit13}, which Nax is based on, extends
the kind structure
with $\{A\,\}\to \kappa$ to support term indices in types.
This extension to kinds allows type constructors with term indicies
such as $\textit{Vec} : * \to \{\textit{Nat}\,\} \to *$ for vectors
(a.k.a. length indexed lists). For instance,
$\textit{Vec}~\textit{Bool}~\{8\}$ is a type of boolean vectors of length 8.
There are two ramifications of term indices being involved in type inference:
\begin{itemize}\vspace*{-.75ex}
\item the unification is modulo equivalence of terms: For instance,
  the type system should consider $\textit{Vec}~\textit{Bool}~\{n\}$ and
  $\textit{Vec}~\textit{Bool}~\{(\lambda x.x)\;n\}$ as equivalent types.
  \vspace*{.5ex}
\item Type inference/checking and kind inference/checking invokes each other:
  A typing rule had to invoke a kinding rule to support
  type constructor polymorphism  (\S\ref{ssec:HMtck}).
  In the extended kind structure, types can appear in kinds
  ($A$ in $\{\!A\}\to\kappa$), therefore, kinding rules
  need to invoke typing rules.
\end{itemize}
Extending our specification with term indicies, while continuing to serve
as a reference implementation, would be an interesting future work that
might involve resolving possible challenges from these two ramifications.

%%%%%% \section{Related Work}\label{sec:relwork} %%%%%%%%%%%%%%%%%%%%%%%%%%
\section{Related Work}\label{sec:relwork}
\subsection{A Framework for Extending HM}
The idea of using logic programming (LP) to specify type inference is not new.
\citet*{HMX99} defined a general framework called \HMX\ for specifying
extensions of HM (e.g., records, type classes, intersection types)
and \citet{tyinferCHR02} implemented \HMX\ using Prolog.
. Testing a type system extension
in the \HMX\ framework provides a certain level of confidence that the extension
would work well with type polymorphism in HM. Testing an extension by
extending our specification provides additional confidence that the extension
would work well with type constructor polymorphism and kind polymorphism,
as well as with type polymorphism.

\subsection{Delayed Goals and Control Flow in Logic Programming}
The concept of delayed goals has been used in many different contexts in LP.
An AILog
% \footnote{A logical inference system for designing intelligent agents.}
	 textbook \cite{AILogTextBook},
introduces delaying goals as one of the useful abilities of a meta-interpreter.
Several Prolog systems including SWI and SICStus provide built-in support for
delaying a goal until certain conditions are met using the predicates
such as {\small\verb|freeze|} or {\small\verb|when|}. In our specification,
%% supporting type constructor polymorphism and kind polymorphism,
we could not simply use {\small\verb|freeze|} or {\small\verb|when|}
because we pre-process the collected delayed goals (see \verb|variablize|
in \S\ref{ssec:HMtck}).
Recently, \citet{SchDemDesWei13} implemented delimited continutations for
Prolog, which might be a useful abstraction for the delayed goals 
used in our work.

\subsection{Other Logic Programming Systems}\label{ssec:otherLP}
Some experimental Prolog implementations support interesting features such as
nominal abstraction in \aProlog\ \cite{cheney04iclp} and a (restricted version
of) higher-order unification in \lProlog\ \cite{nadathur99cade}. However, we
have not found a relational specification of a polymorphic type systems using
them. The \aProlog\ developer attempted to implement the HM type inference for
mini-ML in \aProlog, but failed to produce a working version.\footnote{
	See \texttt{miniml.apl} in the examples directory of
	the \aProlog\ version 0.4 or 0.3.}
The Teyjus \lProlog\ compiler version 2 includes a PCF example,
which is similar to HM but without polymorphic let-bindings.
In both example implementations, they define the type inference predicate
tailored for type inference only (unlike our relational specification that
works for both type checking and inference) and the unification used in
their type inference are manually crafted rather than relying on
the native unification of the LP systems.

Kanren\footnote{ Kanren is a phonetic transcription of the Kanji word
	\begin{CJK}{UTF8}{min}{関連}\end{CJK} meaning ``\emph{relation}''.}
is a delcarative LP system embedded in a pure functional subset of Scheme.
A relational implementation of HM is provided in
the Kanren repository,
% \footnote{See \url{http://kanren.sourceforge.net/\#Sample}}
which works for both type checking,type inference, and also for
type inhabitance searching, as in our HM specification in Prolog.
A simplified version called miniKanren
% \footnote{ \url{http://minikanren.org/} }
has been implemented in several dialects of Scheme and an even further
simplified kernel \muKanren\ \cite{microKanren} is being implemented
in growing number of
host languages as an embedded Domain Specific Language (eDSL) for LP.
By design, Kanren does not provide concrete syntax, therefore, not best suited
for a specification language. However, Kanren has its benefits of being
flexible, simple, and portable. If one is to build a tool based on LP concepts
and wishes to support interfaces to one or more programming languages,
\muKanren\ may be a good choice to target as the backend.

Recently, there has been research on type inference using LP with non-standard
semantics (e.g., corecursive, coinductive) for object-oriented languages
(e.g., featherweight Java) but functional languages were left for
future work \cite{AL-ECOOP09}.
\citet{SRLP15} have developed S-resolution, which is proven \cite{PCR15} to
produce the same results when the depth-first-search style SLD-resolution
used in Prolog terminates, but S-resolution can answer more
queries that makes SLD-resolution loop, because it behaves less eagerly.
%% S-resolution encompasses both inductive
%% and coinductive reasonings by supporting sound productivity checking criteria
%% for recursively defined predicates.
S-resolution might be useful for us to eliminate the need for delayed goals
in our specification.

\subsection{Descriptions of Type Inference Algorithms in ITPs}
There are several formal descriptions of type inference algorithms using
Interactive Theorem Provers (ITPs) such as Coq \cite{Dubois00} and
Isabelle/HOL \cite{UrbanN2009}. The primary motivation %% NaraschewskiN-JAR,
in such work is to formally prove theoretical properties (e.g., soundness,
principal typing) of type inference algorithms, which is different from
our motivation of providing a human readable \& machine executable
specification for the algorithm to reduce the gap between
theoretical specification and practical implementation. Some of those
descriptions  are not even executable because the unification is merely
specified as a set of logical axioms.
Formally describing certifiable type inference in ITPs is challenging
(therefore also challenging to extend or modify) for two reasons. First, fresh
names should be monitored more explicitly and rigorously for the sake of
formal proof. Second, algorithms may need to be massaged differently from
their usual representations, in order to convince the termination checker of
the ITP (e.g. \cite{JFP:185139}).


%%%%%%%%%%%%%%%%%%%%%%%%%%%%%%%%%%%%%%%%%%%%%%%%%%%%%%%%%%%%%%%%%%%%%%%%%%%


\section{Conclusions}\label{sec:concl}

We defined in Prolog a type inference implementation for a non-trivial
type system by making extensive use of Prolog's native support for
unification, allowing us to obtain a concise and runnable
specification. In particular we use logical variables to
represent type (or kind) variables in type (or kind) schemas.

We overcame the conflicting view of type variables between type- and
kind-checking by delaying kind-checking constraints until after
type checking is completed and type variables can be made ground and so
can be used as indexes into the kinding context. We believe this could
have been simplified if there was either a language level support or
well tailored framework for establishing a dual-view on the variables
in logic programming.

We also showed the flexibility of this technique by successively
extending the language with more advanced language features like
recursive let, pattern matching and Mendler-style iteration.

The use of extra-logical built-ins like \verb|copy_term/2| makes so we
can exploit Prolog's logical variables not only for unification but
also to represent binding of quantified type (or kind) variables,
greatly simplifying the handling of implicit polymorphism.

Logical variables can however also limit the predictability of how the
program is going to behave, since a predicate running on inputs that
are not ground enough can get stuck.


\makeatletter
\renewcommand\bibsection{\section*\bibname}
\makeatother
\bibliographystyle{abbrvnat}
\bibliography{main}

\newpage
\section*{Appendix: Bug fixes after submission to APLAS 2015}
Andrea spotted a bug in Fig.~\ref{fig:HMtck}.
In Fig.~\ref{fig:HMtck}, there was a line
in the last \verb|type| predicate rule for mit
\begin{verbatim}
G0 = [kind(KC1,F,K->K), kind(KC1,A->T,o) | G], % delayed goal
\end{verbatim}
Andrea spotted that the latter \verb|KC1| should be \verb|KC| (i.e., without R in it).
After submission, Ki Yung noticed that both \verb|KC1| had to be \verb|KC|.
After this bug fix, kind checking instantiation for type schemes started
to work fine (which didn't at the submission, so we left it out).
Now, we can answer inquiries from reviewers why we didn't have kind checking
in type scheme instantiation: "Well, we fixed the bug right after submission".

Ki Yung just fixed it on in LaTeX file and in the Prolog code
(\verb|HMmicronax.pl|) but have not updated Figures and corresponding text
in the draft with the correct version of type instantiation. Maybe we should
do this after the APLAS review comes in.
The correct instantiation for type scheme should be defined as:
\begin{verbatim}
inst_type(KC,poly(C,T),T1,G,G1) :- copy_term(t(C,T),t(C,T1)),
    free_variables(T,Xs), free_variables(T1,Xs1), zip(Xs,Xs1,Ps),
    findall([kind(KC,X,K), kind(KC,X1,K)],(member((X,X1),Ps),X\==X1),Gs),
    flatten(Gs,G0), append(G0,G,G1).
inst_type(KC,mono(T),T,G,G).

zip([X|Xs],[Y|Ys],[(X,Y)|Zs]) :- zip(Xs,Ys,Zs).
zip([],[],[]).
\end{verbatim}
Instantiation for kind schemes can still use the same \verb|instantiate|
predicate because there are no kind constructor variables, so  all kinds
have sort $\square$.

\end{document}
