\section{Introduction}\label{sec:intro}
When implementing the type system of a programming language, we often face
a gap between the design described on paper and the actual implementation.
Sometimes there is only an ambiguous description in English (or in other
natural languages). Even when there is a mathematical description of
the type system on paper, there can be a gap between the description and
the implementation. Language designers and implementers can suffer from
this gap because it is difficult to determine whether a problem originated
from a flaw in the design or a bug in the implementation.
Having a declarative (i.e., structurally similar to the design)
and flexible (i.e., easily extensible) machine executable specification
is extremely helpful, especially for prototyping or testing
experimental extensions to the language's type system.  
Jones' attempt of \emph{Typing Haskell in Haskell} \cite{JonesTHiH99} is
an exemplary work that demonstrates the value of such a concise, declarative,
and machine executable specification: 90+ pages of Hugs type checker
implementation in C specified in only 400+ lines of readable Haskell.

% In our work, we use Prolog to specify advanced polymorphic features of
% modern functional languages.
Logic programming languages like Prolog are natural candidates for
the purpose of specifying type systems that support type inference:
the syntax and semantics are designed to represent logical inference rules
(which is how type systems are typically formalized) and they offer native
support for unification (which is the basic building block of type inference
algorithms). As a result, type system specifications become even more succinct
in Prolog than in functional languages. More importantly, the specifications
are \emph{relational}, capturing both type checking and type inference
without duplication.

Our contributions are:
\begin{itemize}\vspace*{-1ex}
\item A relational specification that can be executed for
	both purposes of type checking and type inference (\S\ref{ssec:HM}).
\item A succinct and declarative specification for several dimensions of
	polymorphism (type, type constructor, kind) in less than 35 lines
	of Prolog (\S\ref{ssec:HMtck}).
\item An easily extensible specification for pragmatic use
	(as demonstrated in \S\ref{sec:other}).
% We believe our specification is usable without expert knowledge
% in logic programming because we only used built-ins and fairly basic
% library predicates, not relying on any specialized frameworks.
\item Two-staged inference for types and kinds using delayed goals
	(\S\ref{ssec:HMtck}):
We discovered that kind inference can be delayed after type inference
(it is in some sense quite natural) and exploited the fact to get the
most out of Prolog (within its limitations), using Prolog's
native unification as much as possible.
% (which helped our specification to be more succinct).
\item A motivating example of calling support for a dualized
	view on variables in logic programming:
Type variables are viewed as unification variables
in type inference but as concrete/atomic variables
in kind inference in our specification.
Better way of organizing this concept is desirable
to produce better relational specifications for polymorphic type systems.
\end{itemize}

We give a step-by-step tutorial style explanation of our specification in
\S\ref{sec:poly}, gradually extending from the Prolog specification of
the simply-typed lambda calculus. In \S\ref{sec:other}, we demonstrate
that our method of specification is flexible for extensions with
other language features. All our Prolog specification in \S\ref{sec:poly}
and \S\ref{sec:other} are tested on SWI Prolog 7.2 and its source code is
available online.\footnote{
	\url{https://github.com/kyagrd/HMtyInferUsingProlog} }\label{githubURL}
We contemplate on more challenging language features such as GADTs and
term indices in \S\ref{sec:futwork}, discuss related work in
\S\ref{sec:relwork}, and summarize our discussion in \S\ref{sec:concl}.

