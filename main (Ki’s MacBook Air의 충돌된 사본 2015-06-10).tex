% vim: ts=2: sw=2: expandtab: autoindent: spell:
%%%%%%%%%%%%%%%%%%%%%%% file typeinst.tex %%%%%%%%%%%%%%%%%%%%%%%%%
%
% This is the LaTeX source for the instructions to authors using
% the LaTeX document class 'llncs.cls' for contributions to
% the Lecture Notes in Computer Sciences series.
% http://www.springer.com/lncs       Springer Heidelberg 2006/05/04
%
% It may be used as a template for your own input - copy it
% to a new file with a new name and use it as the basis
% for your article.
%
% NB: the document class 'llncs' has its own and detailed documentation, see
% ftp://ftp.springer.de/data/pubftp/pub/tex/latex/llncs/latex2e/llncsdoc.pdf
%
%%%%%%%%%%%%%%%%%%%%%%%%%%%%%%%%%%%%%%%%%%%%%%%%%%%%%%%%%%%%%%%%%%%


\documentclass[runningheads,a4paper]{llncs}

\usepackage{amssymb}
\setcounter{tocdepth}{3}
\usepackage{graphicx}

\usepackage{url}

\newcommand{\keywords}[1]{\par\addvspace\baselineskip
\noindent\keywordname\enspace\ignorespaces#1}

\begin{document}

\mainmatter  % start of an individual contribution

% first the title is needed
\title{Case Study on Polymorphic Type Inference using Prolog}

% a short form should be given in case it is too long for the running head
\titlerunning{Case Study on Polymorphic Type Inference using Prolog}

% the name(s) of the author(s) follow(s) next
%
% NB: Chinese authors should write their first names(s) in front of
% their surnames. This ensures that the names appear correctly in
% the running heads and the author index.
%
\author{Ki Yung Ahn%
%% \thanks{Please note that the LNCS Editorial assumes that all authors have used
%% the western naming convention, with given names preceding surnames. This determines
%% the structure of the names in the running heads and the author index.}%
%% \and Ursula Barth\and Ingrid Haas\and Frank Holzwarth\and\\
%% Anna Kramer\and Leonie Kunz\and Christine Rei\ss\and\\
%% Nicole Sator\and Erika Siebert-Cole\and Peter Stra\ss er
}
%
\authorrunning{Case Study on Polymorphic Type Inference using Prolog}
% (feature abused for this document to repeat the title also on left hand pages)

% the affiliations are given next; don't give your e-mail address
% unless you accept that it will be published
\institute{TODO, TODO,\\
           TODO, TODO \\
     \path|kya@pdx.edu|}

%
% NB: a more complex sample for affiliations and the mapping to the
% corresponding authors can be found in the file "llncs.dem"
% (search for the string "\mainmatter" where a contribution starts).
% "llncs.dem" accompanies the document class "llncs.cls".
%

\toctitle{Case Study on Polymorphic Type Inference using Prolog}
\tocauthor{Case Study on Polymorphic Type Inference using Prolog}
\maketitle


\begin{abstract}
A succinct, declarative, and machine executable specification of
the Hindley--Milner (HM) type inference can be formulated using
logic programming languages such as Prolog. Modern functional
language implementations such as the Glasgow Haskell Compiler
supports more extensive flavors of polymorphism (e.g.,
type constructor polymorphism, nested datatypes, kind polymorphism)
beyond Milner's theory of type polymorphism in late '70s.
In this case study, we progressively extend the Prolog specification
of HM to include support for more advanced type system features.
An interesting development in our series of Prolog specifications is
that extending dimensions of polymorphism resulted in
a multi-staged solution: resolve the typing relations first,
while delaying to resolve kinding relations, and then resolve
the delayed kinding relations. We believe our case study provides
a motivating example for developing theories and tools in logic programming
that provide better support for staged resolution of
different relations at different levels.
\keywords{Hindley--Milner, functional language, type system,
  type inference, unification, parametric polymorphism,
  higher-kinded polymorphism, type constructor polymorphism,
  kind polymorphism, algebraic datatype, nested datatype,
  logic programming, Prolog, delayed goals
%%         GADT,
%%         generalized algebraic datatype
        }
\end{abstract}


\section{Introduction}
TODO
\subsection{TODO}

TODO TODO cite these
%% delayed goals are not uncommon in logic programming
%% (AILog
%% 
%% http://artint.info/html/ArtInt_329.html
%% 
%% https://books.google.com/books?id=Q4i5edW7IyQC&pg=PA321&lpg=PA321

SWI Prolog 7.2

\subsection{TODO}

If you have more than one surname, please make sure that the Volume Editor
knows how you are to be listed in the author index.

\section{Polymorphic type inference specifications in Prolog}

All Prolog specifications commonly start with the following two lines,
which are omitted in Figs.~\ref{fig:STLC}, \ref{fig:HM}, and \ref{fig:HMtck}.
{\small
\begin{verbatim}
  :- set_prolog_flag(occurs_check,true).
  :- op(500,yfx,$).
\end{verbatim}
}\noindent
In the first line, {\small\verb|set_prolog_flag(occurs_check,true)|}
sets Prolog's unification operator \verb|=| to perform occurs check.
For example, {\small\verb|X -> Y = Y|} will fail because of occurs check.
In the second line, {\small\verb|op(500,yfx,$)|} declares {\small\verb|$|}
as a left associative infix operator, which is used to represent
application operator in the object language syntax. For instance,
{\small\verb|E1 $ E2|} is an application of {\small\verb|E1|} to
{\small\verb|E2|}.

\subsection{STLC}
just 4 lines of Prolog
\begin{figure}
\begin{verbatim}
  type(C,var(X),       T) :- first(X:T,C).
  type(C,lam(X,E),A -> B) :- type([X:A|C], E,  B).
  type(C,X $ Y,        B) :- type(C,X,A -> B), type(C,Y,A).

  first(K:V,[K1:V1|Xs]) :- K=K1 -> V=V1 ; first(K:V, Xs).
  \end{verbatim}
\caption{STLC in Prolog}
\label{fig:STLC}
\end{figure}

\subsection{HM}
\begin{figure}
less than 10 lines of Prolog
\begin{verbatim}
  type(C,var(X),      T1) :- first(X:T,C), instantiate(T,T1).
  type(C,lam(X,E),A -> B) :- type([X:mono(A)  |C], E,  B).
  type(C,X $ Y,        B) :- type(C,X,A -> B), type(C,Y,A).
  type(C,let(X=E0,E1), T) :- type([X:mono(A)  |C], E0, A),
                             type([X:poly(C,A)|C], E1, T).
  
  instantiate(poly(CXT,T),T1) :- copy_term(t(CXT,T),t(CXT,T1)).
  instantiate(mono(T),    T).
  
  first(K:V,[K1:V1|Xs]) :- K=K1 -> V=V1 ; first(K:V, Xs).
\end{verbatim}
\caption{HM in Prolog \cite{urlHM}}
\label{fig:HM}
\end{figure}


\subsection{HM + type constructor polymorphism + kind polymorphism}
\begin{figure}
\begin{verbatim}
use_module(library(apply)).
use_module(librar(gensym)).

kind(KC,var(Z),K1) :- first(Z:K,KC), instantiate(K,K1).
kind(KC,F $ G, K2) :- kind(KC,F,K1 -> K2), kind(KC,G,K1).
kind(KC,A -> B,o)  :- kind(KC,A,o), kind(KC,B,o).

type(KC,C,var(X),     T1,G,G ) :- first(X:T,C), instantiate(T,T1).
type(KC,C,lam(X,E), A->B,G,G1) :- type(KC,[X:mono(A)|C],E,B,G0,G1),
                                  G0 = [kind(KC,A->B,o) | G]. % delay goal
type(KC,C,X $ Y,       B,G,G1) :- type(KC,C,X,A->B,G, G0),
                                  type(KC,C,Y,A,   G0,G1).
type(KC,C,let(X=E0,E1),T,G,G1) :- type(KC,[X:mono(A)  |C],E0,A,G, G0),
                                  type(KC,[X:poly(C,A)|C],E1,T,G0,G1).

instantiate(poly(C,T),T1) :- copy_term(t(C,T),t(C,T1)).
instantiate(mono(T),  T).

first(K:V,[K1:V1|Xs]) :- K=K1 -> V=V1 ; first(K:V,Xs).

variablize(var(X)) :- gensym(t,X).

infer_type(KC,C,E,T) :-
  type(KC,C,E,T,[],Gs0), % handle delayed kind sanity check below
  copy_term(Gs0,Gs),
  findall(Ty, member(kind(_,Ty,_),Gs), Tys),
  free_variables(Tys,Xs), maplist(variablize,Xs), % replace free tyvar to var(t)
  findall(A:mono(K),member(var(A),Xs),KC1),       % bindings for new var(t)s
  appendKC(Gs,KC1,Gs1), % extend all KC in kind goals with KC1 for new vars
  maplist(call,Gs1), % run all goals in Gs1

appendKC([],_,[]).
appendKC([kind(KC,X,K)|Gs],KC1,[kind(KC2,X,K)|Gs1]) :- append(KC1,KC,KC2),
                                                       appendKC(Gs,KC1,Gs1).
\end{verbatim}
\caption{HM + type constructor polymorphism + kind polymorphism in Prolog}
\label{fig:HMtck}
\end{figure}



\subsection{Figures}

For \LaTeX\ users, we recommend using the \emph{graphics} or \emph{graphicx}
package and the \verb+\includegraphics+ command.

Please check that the lines in line drawings are not
interrupted and are of a constant width. Grids and details within the
figures must be clearly legible and may not be written one on top of
the other. Line drawings should have a resolution of at least 800 dpi
(preferably 1200 dpi). The lettering in figures should have a height of
2~mm (10-point type). Figures should be numbered and should have a
caption which should always be positioned \emph{under} the figures, in
contrast to the caption belonging to a table, which should always appear
\emph{above} the table; this is simply achieved as matter of sequence in
your source.

Please center the figures or your tabular material
by using the \verb+\centering+
declaration. Short captions are centered by default between the margins
and typeset in 9-point type (Fig.~\ref{fig:example} shows an example).
The distance between text and figure is preset to be about 8~mm, the
distance between figure and caption about 6~mm.

To ensure that the reproduction of your illustrations is of a reasonable
quality, we advise against the use of shading. The contrast should be as
pronounced as possible.

If screenshots are necessary, please make sure that you are happy with
the print quality before you send the files.
\begin{figure}
\centering
\includegraphics[height=6.2cm]{eijkel2}
\caption{One kernel at $x_s$ (\emph{dotted kernel}) or two kernels at
$x_i$ and $x_j$ (\textit{left and right}) lead to the same summed estimate
at $x_s$. This shows a figure consisting of different types of
lines. Elements of the figure described in the caption should be set in
italics, in parentheses, as shown in this sample caption.}
\label{fig:example}
\end{figure}

Please define figures (and tables) as floating objects. Please avoid
using optional location parameters like ``\verb+[h]+" for ``here".

\paragraph{Remark 1.}

In the printed volumes, illustrations are generally black and white
(halftones), and only in exceptional cases, and if the author is
prepared to cover the extra cost for color reproduction, are colored
pictures accepted. Colored pictures are welcome in the electronic
version free of charge. If you send colored figures that are to be
printed in black and white, please make sure that they really are
legible in black and white. Some colors as well as the contrast of
converted colors show up very poorly when printed in black and white.

\subsection{Formulas}

Displayed equations or formulas are centered and set on a separate
line (with an extra line or halfline space above and below). Displayed
expressions should be numbered for reference. The numbers should be
consecutive within each section or within the contribution,
with numbers enclosed in parentheses and set on the right margin --
which is the default if you use the \emph{equation} environment, e.g.,
\begin{equation}
  \psi (u) = \int_{o}^{T} \left[\frac{1}{2}
  \left(\Lambda_{o}^{-1} u,u\right) + N^{\ast} (-u)\right] dt \;  .
\end{equation}

Equations should be punctuated in the same way as ordinary
text but with a small space before the end punctuation mark.

\subsection{Footnotes}

The superscript numeral used to refer to a footnote appears in the text
either directly after the word to be discussed or -- in relation to a
phrase or a sentence -- following the punctuation sign (comma,
semicolon, or period). Footnotes should appear at the bottom of
the
normal text area, with a line of about 2~cm set
immediately above them.\footnote{The footnote numeral is set flush left
and the text follows with the usual word spacing.}

\subsection{Program Code}

Program listings or program commands in the text are normally set in
typewriter font, e.g., CMTT10 or Courier.

\medskip

\noindent
{\it Example of a Computer Program}
\begin{verbatim}
program Inflation (Output)
  {Assuming annual inflation rates of 7%, 8%, and 10%,...
   years};
   const
     MaxYears = 10;
   var
     Year: 0..MaxYears;
     Factor1, Factor2, Factor3: Real;
   begin
     Year := 0;
     Factor1 := 1.0; Factor2 := 1.0; Factor3 := 1.0;
     WriteLn('Year  7% 8% 10%'); WriteLn;
     repeat
       Year := Year + 1;
       Factor1 := Factor1 * 1.07;
       Factor2 := Factor2 * 1.08;
       Factor3 := Factor3 * 1.10;
       WriteLn(Year:5,Factor1:7:3,Factor2:7:3,Factor3:7:3)
     until Year = MaxYears
end.
\end{verbatim}
%
\noindent
{\small (Example from Jensen K., Wirth N. (1991) Pascal user manual and
report. Springer, New York)}

\subsection{Citations}

For citations in the text please use
square brackets and consecutive numbers: \cite{jour}, \cite{lncschap},
\cite{proceeding1} -- provided automatically
by \LaTeX 's \verb|\cite| \dots\verb|\bibitem| mechanism.

\subsection{Page Numbering and Running Heads}

There is no need to include page numbers. If your paper title is too
long to serve as a running head, it will be shortened. Your suggestion
as to how to shorten it would be most welcome.

\section{LNCS Online}

The online version of the volume will be available in LNCS Online.
Members of institutes subscribing to the Lecture Notes in Computer
Science series have access to all the pdfs of all the online
publications. Non-subscribers can only read as far as the abstracts. If
they try to go beyond this point, they are automatically asked, whether
they would like to order the pdf, and are given instructions as to how
to do so.

Please note that, if your email address is given in your paper,
it will also be included in the meta data of the online version.

\section{BibTeX Entries}

The correct BibTeX entries for the Lecture Notes in Computer Science
volumes can be found at the following Website shortly after the
publication of the book:
\url{http://www.informatik.uni-trier.de/~ley/db/journals/lncs.html}

\subsubsection*{Acknowledgments.} The heading should be treated as a
subsubsection heading and should not be assigned a number.

\section{The References Section}\label{references}

In order to permit cross referencing within LNCS-Online, and eventually
between different publishers and their online databases, LNCS will,
from now on, be standardizing the format of the references. This new
feature will increase the visibility of publications and facilitate
academic research considerably. Please base your references on the
examples below. References that don't adhere to this style will be
reformatted by Springer. You should therefore check your references
thoroughly when you receive the final pdf of your paper.
The reference section must be complete. You may not omit references.
Instructions as to where to find a fuller version of the references are
not permissible.

We only accept references written using the latin alphabet. If the title
of the book you are referring to is in Russian or Chinese, then please write
(in Russian) or (in Chinese) at the end of the transcript or translation
of the title.

The following section shows a sample reference list with entries for
journal articles \cite{jour}, an LNCS chapter \cite{lncschap}, a book
\cite{book}, proceedings without editors \cite{proceeding1} and
\cite{proceeding2}, as well as a URL \cite{url}.
Please note that proceedings published in LNCS are not cited with their
full titles, but with their acronyms!

\begin{thebibliography}{4}

\bibitem{jour} Smith, T.F., Waterman, M.S.: Identification of Common Molecular
Subsequences. J. Mol. Biol. 147, 195--197 (1981)

\bibitem{lncschap} May, P., Ehrlich, H.C., Steinke, T.: ZIB Structure Prediction Pipeline:
Composing a Complex Biological Workflow through Web Services. In: Nagel,
W.E., Walter, W.V., Lehner, W. (eds.) Euro-Par 2006. LNCS, vol. 4128,
pp. 1148--1158. Springer, Heidelberg (2006)

\bibitem{book} Foster, I., Kesselman, C.: The Grid: Blueprint for a New Computing
Infrastructure. Morgan Kaufmann, San Francisco (1999)

\bibitem{proceeding1} Czajkowski, K., Fitzgerald, S., Foster, I., Kesselman, C.: Grid
Information Services for Distributed Resource Sharing. In: 10th IEEE
International Symposium on High Performance Distributed Computing, pp.
181--184. IEEE Press, New York (2001)

\bibitem{proceeding2} Foster, I., Kesselman, C., Nick, J., Tuecke, S.: The Physiology of the
Grid: an Open Grid Services Architecture for Distributed Systems
Integration. Technical report, Global Grid Forum (2002)

\bibitem{url} National Center for Biotechnology Information, \url{http://www.ncbi.nlm.nih.gov}

\bibitem{urlHM} Andrea Vezzosi, Hindley Milner in 7 lines of hacky prolog, \url{http://lpaste.net/65035|}

\end{thebibliography}


\section*{Appendix: Springer-Author Discount}

LNCS authors are entitled to a 33.3\% discount off all Springer
publications. Before placing an order, the author should send an email, 
giving full details of his or her Springer publication,
to \url{orders-HD-individuals@springer.com} to obtain a so-called token. This token is a
number, which must be entered when placing an order via the Internet, in
order to obtain the discount.

\section{Checklist of Items to be Sent to Volume Editors}
Here is a checklist of everything the volume editor requires from you:


\begin{itemize}
\settowidth{\leftmargin}{{\Large$\square$}}\advance\leftmargin\labelsep
\itemsep8pt\relax
\renewcommand\labelitemi{{\lower1.5pt\hbox{\Large$\square$}}}

\item The final \LaTeX{} source files
\item A final PDF file
\item A copyright form, signed by one author on behalf of all of the
authors of the paper.
\item A readme giving the name and email address of the
corresponding author.
\end{itemize}
\end{document}
