\section{Related Work}\label{sec:relwork}
\subsection{A framwrowk for extending HM}
The idea of using logic programming to specify type inference is not new.
\citet*{HMX99} defined a general framework called \HMX\ for specifying
extensions of HM (e.g., records, type classes, intersection types)
and \citet{tyinferCHR02} implemented \HMX\ using Prolog with
constraint handling rules (CHR). Testing a type system extension
in the \HMX\ framework provides a certain level of confidence that the extension
would work well with type polymorphism in HM. Testing an extension by
extending our specification provides additional confidence that the extension
would work well with type constructor polymorphism and kind polymorphism,
as well as with type polymorphism.

\subsection{Delayed Goals and Control Flow in Logic Programming}
The concept of delayed goals have been used in many different contexts
in logic programming. An AILog\footnote{A logical inference system for designing
	 intelligent agents.} textbook \cite{AILogTextBook},
introduces delaying goals as one of the useful abilities of meta-interpreter.
Many Prolog systems, such as SWI or SICStus, provide built-in support for
delaying a goal until certain conditions are met using the predicates
such as \verb|freeze| or \verb|when|. In our specification supporting
type constructor polymorphism and kind polymorphism, we could not
simply use \verb|freeze| or \verb|when| because we pre-process
the collected delayed goals (see \verb|variableze| in \S\ref{ssec:HMtck}).
Recently, \citet{SchDemDesWei13} implemented delimited continutations for Prolog,
which might be a usful abstraction for describing the delayed goals used in
our specification.

\subsection{Other Logic Programming systems}
There are experimental Prolog implementations that support peculiar features
such as \aProlog\ \cite{cheney04iclp} supporting nominal abstraction in a purely
logical setting and \lProlog\ \cite{nadathur99cade} supporting (restricted
version of) higher-order unification. Such features may help us specify universal
quantifications and fresh name generations more elegantly, but there is a
trade-off for not having the pragmatic support (e.g. built-ins for extra-logical
features, documentation, and graphical debugging) of more widely-used and
well-maintained systems such as SWI or SICStus. In fact, the developer of
\aProlog\ attempted to implement of HM type inference for mini-ML in \aProlog,
but failed to produce a working version.\footnote{See \texttt{miniml.apl}
	in the examples directory of the \aProlog\ version 0.4 or 0.3.}
The Teyjus \lProlog\ compiler version 2 distribution includes an example of
a PCF implementation in \lProlog, whose polymorphic type inference is similar
to HM but without polymorphic let-bidings. Both the borken mini-ML implementation
in \aProlog\ and the working PCF implementation \lProlog\ defines their
type inference predicate tailored for type inference only, unlike our relational
description that works for multiple purposes (type checking \& inferernce),
and the unification used in the mini-ML and PCF implementations are user defined,
rather than relying on the native unification provided by the logic programming
systems.

Recently, there has been research on type inference using logic programming
with non-standard semantics (e.g., corecursive, coinductive, or coalgebraic)
for object-oriented languages (e.g., featherweight Java) but
functional languages were left for future work \cite{AL-ECOOP09}.
\TODO{talk about Structural Resolution TODO what to cite?}

Kanren\footnote{Kanren is a phonetic transcription of the Kanji word
	\begin{CJK}{UTF8}{min}{関連}\end{CJK} meaning ``\emph{relation}''.}
is a delcarative logic programming system embedded in a pure functional subset of
Scheme. A relational implementation of HM is provided in
the Kanren repository,\footnote{See \url{http://kanren.sourceforge.net/\#Sample}}
which works for both type checking and type infernece, and also for
type inhabitance searching, as in our HM specification in Prolog.
A simplified version of Kanren called miniKANREN\footnote{
	\url{http://minikanren.org/} } has been implemented in several dialects
of Scheme and even further simplified kernel \muKanren\ (microKanren)
\cite{microKanren} are being implemeted in growing number of host languages as
embeded Domain Specific Language (eDSL) for logic programming. By design, Kanren
and its simplified versions do not provide concrete syntax but rely on the syntax
of their host language. Therefore, Kanren itself is not best suited for
a speicifcation language, but has its benifits of being flexibile, simple, and
portable. If one is to build a tool based on logic programming concepts and wants
to support interfaces to one ore even more programming languages, \muKanren\ may
be a good choice to targat as the backend.

\subsection{Descriptions of Type Inference Algorithms in ITPs}
There are several formal descriptions of type inference algorithms using
Interactive Theorem Provers (ITPs) such as Coq \cite{Dubois00} and
Isabelle/HOL \cite{NaraschewskiN-JAR,UrbanN2009}. The primoary motivation
in such work is to formally prove theoretical properties (e.g., soundness,
principal typing) of type inference algorithms, which is differeent from
our motivation of providing a human readable \& machine executable
specificiation for the algorithm to reduce the gap between
theoretical specification and practical implementation. Some of those
descriptions  are not even executable because the unification is merely
specified as set of logical axioms, only for the sake of proving properties.
Formally describing certifiable type inference algorhthms ITPs are challenging
(therefore also challenging to extend or modify) for two reasons. First, fresh
name geneartion should be monitored more explicitly and rigourously in order to
provide proper information for the sake of formal proof. Second, some parts of
the algorithm may need to be massaged into certain froms, differrent from their
usual represnetations, in order to convince the termination checker of the ITP
(e.g. \cite{JFP:185139}).

