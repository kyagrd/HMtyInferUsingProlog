\section{Related Work}\label{sec:relwork}
The idea of using logic programming to specify type inference is not new.
\citet*{HMX99} defined a general framework called \HMX\ for specifying
extensions of HM (e.g., records, type classes, intersection types)
and \citet{tyinferCHR02} implemented \HMX\ using Prolog with
constraint handling rules (CHR). Testing a type system extension
in the \HMX\ framework provides a certain level of confidence that the extension
would work well with type polymoprhism in HM. Testing an extension by
extending our specification provides additional confidence that the extesion
would work will with type constructor polymorphism and kind polymorphism,
as well as with type polymorhpism.

The concept of delayed goals have been used in many differrent context
in logic programming. An AILog\footnote{A logical inference system for designing
	 intelligent agents.} textbook \cite{AILogTextBook},
introduces delaying golas as one of the useful abilities of meta-interpreter.
Many prolog systems, such as SWI or SICStus, provide built-in support for
delaying a goal untill certiain conditions are met using the predicates
such as \verb|freeze| or \verb|when|. In our implementation of
type constructor polymorphism and kind polymorphsim, we cannot
simiply use freeze or when because we need pre-processng on
the collected delayed goals (see \verb|variableze| in \S\ref{ssec:HMtck}).

%% delayed goals are not uncommon in logic programming
%% http://artint.info/html/ArtInt_329.html
%% 
%% https://books.google.com/books?id=Q4i5edW7IyQC&pg=PA321&lpg=PA321
%%
%% https://books.google.com/books?id=d2_i0qeP6yUC&pg=PA645&lpg=PA645

TODO
$\lambda$Prolog 
$\alpha$Prolog 
has certain features that support certain concepts in a purely logical manner
but flexibility availability trade-off
SWI or SICStus
TODO\\

Recently, there has been research on type inference using logic programming
with non-standard semantics (e.g., corecursive, coinductive, or coalgebraic)
(e.g., \cite{})
featherweight java

